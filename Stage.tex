\documentclass[12pt,a4paper,twoside]{article}

\usepackage{amsmath}
\usepackage{amsfonts}
\usepackage{amssymb}
\usepackage{graphicx}
\usepackage{caption}
\usepackage{fancyhdr}
\usepackage{lastpage}
\usepackage{wrapfig}
\usepackage{color}
\usepackage{fancybox}
\usepackage[T1]{fontenc}
% \usepackage{hangcaption}
\usepackage{listings}
\usepackage[utf8]{inputenc}
\usepackage[francais]{babel}
\usepackage{parskip}
\usepackage{float}
\usepackage{subcaption}
\usepackage{subfig}

\usepackage{enumitem}
\newcommand{\hsp}{\hspace{20pt}}
\newcommand{\HRule}{\rule{\linewidth}{0.5mm}}

\usepackage[a4paper,left=2cm,right=2cm,top=2cm,bottom=2cm]{geometry}

\geometry{
margin=2.5cm,
includeheadfoot,
}

%\usepackage{libertine}
\author{Antoine Roche}
\title{Rapport de Stage}

\begin{document}

    \begin{titlepage}                                    %%% PAGE DE GARDE %%%
        \begin{sffamily}
            \begin{center}

                % Upper part of the page. The '~' is needed because \\
                % only works if a paragraph has started.

                %% Les deux images cote à cote
                \begin{figure}[h]
                    \begin{minipage}[c]{.46\linewidth}
                        \centering
                        \includegraphics[scale=0.2]{ressources/images/logos/Logo_Reims_University.png}
                        %\caption*{Cea}
                    \end{minipage}
                    \hfill%
                    \begin{minipage}[c]{.46\linewidth}
                        \centering
                        \includegraphics[scale=0.1]{ressources/images/logos/logo_cea.png}
                    \end{minipage}
                \end{figure}
                \vspace{2cm}
                \textsc{\LARGE Université de Reims Champagne-Ardenne}\\[2cm]
                \textsc{\Large Rapport de stage Master 1}\\[1.5cm]
                % Title
                \HRule \\[1cm]
                { \Large \bfseries Etude et évaluation de la structure de donnée SVDAG et ses variantes pour le RayTracing en visualisation scientifique \\[0.4cm] }
                \HRule \\[2cm]

                % Author and supervisor

                \begin{minipage}{0.8\textwidth}
                    \begin{flushleft}
                        \Large Auteur: \textsc{Antoine Roche}\\
                        Tuteur de stage: \textsc{Jérôme Dubois}\\
                        Tuteur enseignant : \textsc{Mickael Krajecki}\\
                    \end{flushleft}
                \end{minipage}

                \vfill
                \Large 08/04/2019 — 30/08/2019 \\[1cm]
                \Huge {CEA, DAM, DIF, F-91297, Arpajon, France}
                % Bottom of the page

            \end{center}
        \end{sffamily}
    \end{titlepage}


    \newpage

    \renewcommand{\contentsname}{Sommaire}

    \tableofcontents


    \newpage

    \lstset{numbers=left, tabsize=3, frame=single, numberstyle=\ttfamily,
    basicstyle=\footnotesize}
    \thispagestyle{empty}

    \section{Présentation CEA}                              %%% PRESENTATION CEA %%%

    Acteur majeur de la recherche, du développement et de l'innovation, le Commissariat à l’énergie atomique et aux énergies alternatives intervient dans quatre domaines :

    \begin{itemize}[label=\textbullet]
        \item    La défense et la sécurité. ;
        \item    Les énergies bas carbone (nucléaires et renouvelables). ;
        \item    La recherche technologique pour l’industrie. ;
        \item    La recherche fondamentale (sciences de la matière et sciences de la vie).
    \end{itemize}

    S’appuyant sur une capacité d’expertise reconnue, le CEA participe à la mise en place de projets de collaboration
    avec de nombreux partenaires académiques et industriels. Le CEA est implanté sur 9 centres répartis dans toute la
    France. Il développe de nombreux partenariats avec les autres organismes de recherche, les collectivités locales
    et les universités. A ce titre, le CEA est partie prenante des alliances nationales coordonnant la recherche
    française dans les domaines de l’énergie (ANCRE), des sciences de la vie et de la santé (AVIESAN), des sciences
    et technologies du numérique (ALLISTENE), des sciences de l’environnement (AlIEnvi) et des sciences humaines et
    sociales (ATHENA).

    Reconnu comme un expert dans ses domaines de compétence, le CEA est pleinement inséré dans l'espace européen de
    la recherche et exerce une présence croissante au niveau international. Le CEA compte 15 942 techniciens,
    ingénieurs, chercheurs et collaborateurs pour un budget de 5 milliards d'euros (chiffres publiés fin 2017).

    \begin{figure}[b]
        \centering
        \makebox[\textwidth]{
        \raisebox{-80pt}[0pt][70pt]{
        \includegraphics[width=0.8\textwidth]{ressources/diagrammes/organigramme_cea.png}
        }
        }
    \end{figure}

    \newpage

    \section*{La direction des applications militaires}

    \subsection*{Une direction au service de la dissuasion}
    La Direction des applications militaires (DAM) du CEA, a pour mission
    de concevoir, fabriquer, maintenir en condition opérationnelle, puis démanteler
    les têtes nucléaires qui équipent les forces nucléaires aéroportée et océanique
    françaises.

    La DAM est chargée de la conception et de la réalisation des réacteurs et de
    c\oe urs nucléaires équipant les bâtiments de la Marine nationale, sous-marins
    et porte-avions. Elle apporte son soutien à la Marine nationale pour le suivi en
    service et le maintien en condition opérationnelle de ces réacteurs.

    La DAM est également responsable de l'approvisionnement des matières nucléaires
    straté\-giques pour les besoins de la dissuasion.

    Dans un monde en profonde mutation, la DAM contribue aussi à la sécurité
    nationale et
    \begin{wrapfigure}[12]{r}{10cm}
        \includegraphics[width=10cm]{ressources/images/dam/5_thumbnails.jpg}
    \end{wrapfigure}
    internationale à travers l'appui technique qu'elle apporte aux autorités, pour
    les questions de lutte contre la prolifération nucléaire et le terrorisme et de
    désarmement.

    Depuis le transfert du centre de Gramat en 2010 de la Direction générale de
    l'armement au CEA, la DAM apporte son expertise à la Défense dans le domaine de
    l'armement conventionnel.

    \subsection*{Une direction ouverte à la recherche}
    Le partage national et international des connaissances (lorsqu'il est possible),
    la confrontation à l'évaluation scientifique extérieure, l'intégration à des
    réseaux de compétences constituent des gages de crédibilité scientifique.

    Les équipes de la DAM réalisent chaque année environ 2000 publications et
    communications scientifiques. Cette ouverture de la DAM passe également par la
    mise à la disposition de la communauté des chercheurs de ses moyens
    expérimentaux et par la contribution de ses équipes à d'autres programmes de
    recherche.

    \subsection*{Une direction actrice de la politique industrielle française}
    La DAM partage très largement son activité avec l'industrie française : c'est
    ainsi que le montant des achats, auprès de celle-ci, représente plus des deux
    tiers de son budget ; le dernier tiers se répartit entre les salaires des
    personnels (un cinquième) et les taxes.

    La politique industrielle de la DAM est originale à plus d'un titre :

    \begin{itemize}[label=\textbullet]
        \item
        d'abord parce que la DAM conserve la maîtrise d'\oe uvre
        d'ensemble de la grande majorité des systèmes dont elle a la
        responsabilité : elle veille ainsi au juste équilibre entre les
        grands groupes industriels de la Défense et les PME souvent
        innovantes, en contractualisant directement avec ces dernières,
        leur permettant ainsi de recevoir la juste rémunération de leur
        production ;
        \item
        ensuite, parce que la répartition de son budget est sous-tendue
        par une répartition des travaux : la DAM conduit la recherche
        dans ses laboratoires grâce à son personnel de haut niveau
        scientifique et technologique. Une fois la définition d'un
        produit acquise, la DAM transfère la définition et les procédés
        vers les industriels qui en réalisent le développement, puis la
        production.
    \end{itemize}

    La DAM a également pour objectif que ses centres participent à la vie économique
    locale par leur implication dans les pôles de compétitivité. Hors de son propre
    champ d'utilisation, elle valorise ses recherches par le transfert de
    technologies vers l'industrie et le dépôt de nombreux brevets.

    \subsection*{Le format}
    La DAM comprend cinq centres aux missions homogènes, dont les activités se
    répartissent entre la recherche de base, le développement et la fabrication :
    \begin{figure}[!ht]
        \begin{minipage}{0.6\linewidth}
            \begin{itemize}[label=\textbullet]
                \item
                {\bf DAM Ile-de-France (DIF)}, à Bruyères-le-Châtel, où sont
                menés les travaux de physique des armes, les activités de
                simulation numérique et de lutte contre la prolifération
                nucléaire ; DIF est aussi le centre responsable de l'ingénierie
                à la DAM ; enfin, au centre DIF est rattachée l'INBS-Propulsion
                Nucléaire du centre CEA/Cadarache, en région Provence Alpes-Côte
                d'Azur, où sont implantées les installations d'essais à terre et
                une partie des fabrications de la propulsion nucléaire ;
            \end{itemize}
        \end{minipage}
        \begin{minipage}{0.4\linewidth}
            \centering
            \includegraphics[width=0.7\textwidth]{ressources/images/dam/5_centres.png}
        \end{minipage}
    \end{figure}

    \begin{itemize}[label=\textbullet]
        \item
        {\bf Le Cesta}, en Aquitaine, consacré à l'architecture des armes, aux
        tests de tenue à l'environnement. Il met en oeuvre le Laser Mégajoule,
        équipement majeur de la Simulation ;
        \item
        {\bf Valduc}, en Bourgogne, dédié aux matériaux nucléaires et à
        l'installation expérimentale Epure du programme Simulation ;
        \item
        {\bf Le Ripault}, en région Centre, dédié aux matériaux non nucléaires
        (explosifs chimiques\textellipsis) ;
        \item
        {\bf Gramat}, (ex-DGA) en Midi-Pyrénées, qui conduit au profit de la
        Défense des activités en vulnérabilité des systèmes et efficacité des
        armements. ;
    \end{itemize}



    \newpage
    \section*{Le centre DAM Ile-de-France}
    \begin{figure}[H]
        \centering
        \includegraphics[scale=0.4]{ressources/images/dam/vue_aerienne.png}
        \caption*{Centre DAM Île-de-France}
    \end{figure}

    Le CEA/DAM - Île de France (DIF) est l'une des directions opérationnelles de la DAM. Le site de la DIF compte
    environ 2000 salariés CEA et accueille quotidiennement environ 600 salariés d'entreprises extérieures. Il est
    situé à Bruyères-le-Châtel à environ 40 km au sud de Paris, dans l'Essonne.

    Les missions de la DIF comprennent :

    \begin{itemize}[label=\textbullet]
        \item
        La conception et garantie des armes nucléaires, grâce au programme Simulation. L'enjeu consiste à reproduire par le calcul les différentes phases du fonctionnement d'une arme nucléaire et à confronter ces résultats aux mesures des tirs nucléaires passés et aux résultats expérimentaux obtenus sur les installations actuelles (machine radiographique, lasers de puissance, accélérateurs de particules). ;
        \item
        La lutte contre la prolifération et le terrorisme, en contribuant notamment au programme de garantie du Traité de Non-Prolifération et en assurant l'expertise technique française pour la mise en œuvre du Traité d'Interdiction Complète des Essais Nucléaires (TICE).
        \item
        L'expertise scientifique et technique, dans le cadre de la construction et du démantèlement d'ouvrages complexes ainsi que pour la surveillance de l'environnement et les sciences de la terre.
        \item
        L'alerte des autorités, mission opérationnelle assurée 24h sur 24, 365 jours par an, en cas d'essai nucléaire, de séisme en France ou à l'étranger, et de tsunami dans la zone Euro-méditerranéenne. La DIF fournit aux autorités les analyses et synthèses techniques associées.

    \end{itemize}

    Depuis 2003, le centre DAM-Île-de-France héberge le complexe de calcul scientifique du CEA, qui regroupe l’ensemble des supercalculateurs du CEA, et qui comprend à ce jour :

    \begin{itemize}[label=\textbullet]
        \item
        Le supercalculateur Tera1000-1 pour les besoins du programme Simulation du CEA/DAM, mis en service en 2016, dispose d’une puissance de calcul de 2,5 petaflops, c’est à dire capable d’effectuer 2,5 millions de milliards d’opérations par seconde. Il est complété en 2018 par Tera1000-2, autre composante du projet Tera1000, qui préfigure les architectures et technologies du futur supercalculateur qui sera installé à l’horizon 2020.Sa puissance de calcul est de 12,5 petaflops.

        \begin{figure}[H]
            \centering
            \includegraphics[scale=1]{ressources/images/dam/tera1000.jpg}
        \end{figure}

        \item
        Le supercalculateur Cobalt du Centre de Calcul pour la Recherche et la Technologie (CCRT), ouvert à la communauté civile de la recherche et de l’industrie, pour une puissance globale de 1,5 petaflops.
        \begin{figure}[H]
            \centering
            \includegraphics[scale=1]{ressources/images/dam/cobalt.jpg}
        \end{figure}

        \item
        Le supercalculateur IRENE, d’une puissance de 9 petaflops, deuxième élément d’un réseau de supercalculateurs de classe petaflopique destiné aux chercheurs de la communauté scientifique européenne. Ce supercalculateur est hébergé au TGCC (Très Grand Centre de Calcul) et exploité par les équipes du CEA, qui apporte ainsi sa contribution à la participation de la France au projet PRACE (Partnership for Advanced Computing in Europe)

    \end{itemize}



    \newpage


    \section{Introduction}                          %%% INTRODUCTION %%%

    Le domaine de la visualisation, que ce soit dans le secteur scientifique ou multimédia, nécessite une qualité de rendu toujours plus grande.


    \newpage
    \section{Présentation du stage}                 %%% PRESENTATION STAGE %%%


    \newpage
    \section{Structures de données}                 %%% STRUCTURES DONNEES %%%
    \subsection{Sparse Voxel Octree}

    Le Sparse Voxel Octree (SVO) est une représentation de données en arbre. Sa structure permet d'accéder à une information rapidement.
    Il est construit à partir d'une scène 3D non structurée, qui est ensuite découpée en grille dont les cases contenant un élément sont à leur tour découpées jusqu'à atteindre le niveau défini.
    Chaque feuille du dernier niveau devient alors un voxel.

    \begin{figure}[H]
        \centering{
        \resizebox{90mm}{!}{\input{ressources/inkscape/SVO1.pdf_tex} \space \space \input{ressources/inkscape/SVO2.pdf_tex}}
        \caption{Découpage d'une scène 2D avec 3 niveaux et représentation des "voxels" obtenus}
        \label{fig:svo1}
        }
    \end{figure}


    \begin{figure}[H]
        \centering{
        \resizebox{150mm}{!}{\input{ressources/inkscape/SVO3.pdf_tex}}
        \caption{Représentation de l'arbre construit}
        \label{fig:svo3}
        }
    \end{figure}

    \subsection{Sparse Voxel Directed Acyclic Graph}

    Le Sparse Voxel Directed Acyclic Graph (SVDAG) est un arbre de voxel compressé. Il peut être construit à partir d'un SVO.
    Sa particularité est qu'un noeud peut être pointé par plusieurs parents, évitant ainsi de stocker plusieurs noeuds s'ils sont identiques.
    Il est également acyclic, nous pouvons descendre dans l'arbre mais pas remonter. Cela offre un gain de stockage supplémentaire en ne spécifiant pas les parents des noeuds.
    Tout cela permet de faire tenir en mémoire des structures trop volumineuses à la base et de rendre les accès mémoires plus rapides.

    \begin{figure}[H]
        \centering{
        \resizebox{80mm}{!}{\input{ressources/inkscape/svdag1.pdf_tex}}
        \caption{Représentation d'un arbre de données}
        \label{fig:svdag1}
        }
    \end{figure}

    \begin{figure}[H]
        \centering{
        \resizebox{35mm}{!}{\input{ressources/inkscape/svdag2.pdf_tex}}
        \caption{Représentation de l'arbre après compression via SVDAG}
        \label{fig:svdag2}
        }
    \end{figure}
    \newpage
    \subsection{Adaptive Mesh Refinement}

    L'Adaptive Mesh Refinement (AMR) est un arbre semblable au SVO, à la grande différence que les voxels peuvent se situer à des niveaux différents dans l'arbre.
    Ceci implique qu'une scène peut contenir des voxels plus grossiers que d'autres si un affinage n'est pas nécessaire.
    En contrepartie, des zones peuvent être très détaillées sans allourdir la mémoire de façon excessive. \\
    De plus, il est aisé de diminuer le niveau maximal des voxels pour obtenir une scène moins détaillée
    mais beaucoup plus fluide (voire en temps réel) afin de configurer rapidement les différents paramètres pour le rendu final.

    \begin{figure}[H]
        \centering{
        \resizebox{100mm}{!}{\input{ressources/inkscape/amr_grille1.pdf_tex} \space \space \input{ressources/inkscape/amr_arbre1.pdf_tex}}
        \caption{Grille de type AMR de niveau 3 avec son arbre associé}
        \label{fig:amr1}
        }
    \end{figure}

    \begin{figure}[H]
        \centering{
        \resizebox{100mm}{!}{\input{ressources/inkscape/amr_grille2.pdf_tex} \space \space \input{ressources/inkscape/amr_arbre2.pdf_tex}}
        \caption{Grille de type AMR de niveau 2 avec son arbre associé}
        \label{fig:amr2}
        }
    \end{figure}


    \newpage
    \section{Performance et analyse du SVDAG}

    \begin{figure}[H]
        \centering
        \resizebox{160mm}{!}{\includegraphics[width=0.2\linewidth]{ressources/images/bunny.png}
        \includegraphics[width=0.2\linewidth]{ressources/images/dragon.png}
        \includegraphics[width=0.2\linewidth]{ressources/images/hairball.png}
        \includegraphics[width=0.2\linewidth]{ressources/images/lucy.png}}
        \caption{Rendu de modèles avec du raycasting sur SVDAG}
        \begin{center}
            De gauche à droite : Bunny (Stanford), Dragon (Stanford), Hairball (NVIDIA Research), Lucy (Stanford)
        \end{center}
        \label{models}
    \end{figure}




    \begin{table}[H]
        \begin{tabular}{lllll}
            \hline
            \multicolumn{1}{|l|}{}         & \multicolumn{1}{l|}{\begin{tabular}[c]{@{}l@{}}
                                                                     Nombre\\ Triangles
            \end{tabular}} & \multicolumn{1}{l|}{\begin{tabular}[c]{@{}l@{}}
                                                     Taille\\ non-structuré
            \end{tabular}}             & \multicolumn{1}{l|}{\begin{tabular}[c]{@{}l@{}}
                                                                 Temps construction\\ SVO
            \end{tabular}}             & \multicolumn{1}{l|}{\begin{tabular}[c]{@{}l@{}}
                                                                 Temps constrution\\ SVDAG
            \end{tabular}} \\ \hline
            \multicolumn{1}{|l|}{Bunny}    & \multicolumn{1}{l|}{69 451}                                                     & \multicolumn{1}{l|}{3.0 Mo}                                                                     & \multicolumn{1}{l|}{3.9s}                                                                         & \multicolumn{1}{l|}{4.3s}                                                              \\ \hline
            \multicolumn{1}{|l|}{Dragon}   & \multicolumn{1}{l|}{871 414}                                                    & \multicolumn{1}{l|}{33.8 Mo}                                                                    & \multicolumn{1}{l|}{5.7s}                                                                         & \multicolumn{1}{l|}{3.0s}                                                              \\ \hline
            \multicolumn{1}{|l|}{Hairball} & \multicolumn{1}{l|}{2 880 000}                                                  & \multicolumn{1}{l|}{236.1 Mo}                                                                   & \multicolumn{1}{l|}{58.7s}                                                                        & \multicolumn{1}{l|}{47.7s}                                                             \\ \hline
            \multicolumn{1}{|l|}{Lucy}     & \multicolumn{1}{l|}{28 055 742}                                                 & \multicolumn{1}{l|}{533.1 Mo}                                                                   & \multicolumn{1}{l|}{77.5s}                                                                        & \multicolumn{1}{l|}{1.7s}                                                              \\ \hline
            & & & &                                                                                        \\ \hline
            \multicolumn{1}{|l|}{}         & \multicolumn{1}{l|}{\begin{tabular}[c]{@{}l@{}}
                                                                     Nombre\\ voxels
            \end{tabular}}    & \multicolumn{1}{l|}{\begin{tabular}[c]{@{}l@{}}
                                                        Taille SVO\\ Structure + grandeurs
            \end{tabular}} & \multicolumn{1}{l|}{\begin{tabular}[c]{@{}l@{}}
                                                     Taille SVDAG\\ Structure + grandeurs
            \end{tabular}} & \multicolumn{1}{l|}{Taux compression}                                                  \\ \hline
            \multicolumn{1}{|l|}{Bunny}    & \multicolumn{1}{l|}{3 591 666}                                                  & \multicolumn{1}{l|}{14.4 + 57.5 Mo}                                                             & \multicolumn{1}{l|}{1.3 + 8.5 Mo}                                                                 & \multicolumn{1}{l|}{86.3\%}                                                            \\ \hline
            \multicolumn{1}{|l|}{Dragon}   & \multicolumn{1}{l|}{2 688 970}                                                  & \multicolumn{1}{l|}{10.8 + 43.0 Mo}                                                             & \multicolumn{1}{l|}{1.0 + 6.4 Mo}                                                                 & \multicolumn{1}{l|}{86.2\%}                                                            \\ \hline
            \multicolumn{1}{|l|}{Hairball} & \multicolumn{1}{l|}{41 521 450}                                                 & \multicolumn{1}{l|}{166.1 + 664.3 Mo}                                                           & \multicolumn{1}{l|}{13.3 + 94.6 Mo}                                                               & \multicolumn{1}{l|}{87.0\%}                                                            \\ \hline
            \multicolumn{1}{|l|}{Lucy}     & \multicolumn{1}{l|}{1 540 004}                                                  & \multicolumn{1}{l|}{6.2 + 24.3 Mo}                                                              & \multicolumn{1}{l|}{0.6 + 3.6 Mo}                                                                 & \multicolumn{1}{l|}{86.3\%}                                                            \\ \hline
        \end{tabular}
        \caption{Résultats avec des scènes voxélisées de résolution 1K$^{3}$}
    \end{table}



    \newpage
    \section{Collaboration et contribution}

    \newpage
    \section{Problèmes rencontrés}

    \newpage
    \section{Travaux futurs}

    \newpage
    \section{Conclusion}                            %%% CONCLUSION %%%

    \newpage
    \section{Remerciements}                         %%% REMERCIEMENTS %%%
    Obligatoire dans le rapport

    \newpage
    \section{Glossaire}                             %%% GLOSSAIRE %%%
    Obligatoire dans le rapport


    \newpage
    \section{Bibliographie}                         %%% BIBLIOGRAPHIE %%%
    Obligatoire (il existe des solutions pour générer automatiquement les bib)


\end{document}